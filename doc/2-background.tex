\section{Background and Related Work}
\label{sec:background_and_related_work}

Separation logic is an extension of Floyd–Hoare logic, that has been around since 2002. \cite{reynolds_separation_2002} There have been many innovations related to it since then. Oracle semantics for concurrent separation logic \cite{drossopoulou_oracle_2008}, separation algebra for a language called concurrent C minor \cite{hutchison_fresh_2009},  Multimodal, and higher order separation logics as well. \cite{dockins_multimodal_2008,noauthor_higher_nodate} Especially in the last 4 or so years, there has been enough progress in the subject to warrant a paper to organize this "jungle" of separation logics. \cite{chang_bringing_2017} Fairly recent implementations have made it into VST (Verified Tool Chain), which is coded in the Coq programming language and proof management system. \cite{hutchison_verified_2012} VST is a tool chain that is being developed mostly for the verification and compilation of the C programming language. VST has many modules, all of which use some separation logic to some extent. 

The first major chunk is a specification, a front end language called Verifiable C \cite{noauthor_verifiable_nodate}, which uses higher-order impredicative concurrent separation logic. Verifiable C works as a logical language that is mostly a restriction of the C language. Much like python 3 has what are called type hints \cite{noauthor_typing_nodate}, Verifiable C has constructs that help the underlying tool chain make more sense of C programs and verify their correctness. The next large section of code takes Verifiable C and checks for correctness. This part has parts called VST-FLoyd and a few other components. VST-Floyd is part of the core tactics of Verified C, and makes proving safety and correctness of C programs much more automatic. This section of code seems to have the ability to work outside of the C program verifying toolchain to also do separation logic operations as a Coq library. \cite{noauthor_princetonuniversityvst_2021}

On the back end is the last major chunk called CompCert, a C compiler that also uses separation logic in its compilation process. \cite{noauthor_compcert_nodate} This is not open source software and is owned by INRIA. It is a software project that is dual licensed. So I was not sure if contributions could be made to the software that would be welcome.

To understand this software stack I have found a series called Software Foundations. Software Foundations is a book series on solving, verifying, and some synthesis of programs with the Coq language. \cite{noauthor_software_nodate} There are 5 volumes covering these concepts. The most important for this project is volume 5: Verifiable C. There are also many manuals for parts of VST to go through, as needed. But it turns out the most useful way to learn VST and the Coq language was to look at the manuals for both languages. Software foundations is rated to take a semester for the first book, and potentially that long for each subsequent book. There was simply not enough time in the semester for this much reading.
