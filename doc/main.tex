\documentclass[sigconf]{acmart}
%%
%% \BibTeX command to typeset BibTeX logo in the docs
\AtBeginDocument{%
  \providecommand\BibTeX{{%
    \normalfont B\kern-0.5em{\scshape i\kern-0.25em b}\kern-0.8em\TeX}}}

%% These commands are for a PROCEEDINGS abstract or paper.
\setcopyright{none}
\settopmatter{printacmref=false}
\renewcommand\footnotetextcopyrightpermission[1]{}
\acmConference[ECE595SE]{ECE595: Advanced software engineering}{February 2021}{West Lafayette, IN, USA}
\acmPrice{00.00}
\acmISBN{978-1-4503-XXXX-X/18/06}

\usepackage[
    backend=biber,
    style=ieee,
  ]{biblatex}
\addbibresource{references.bib}
\usepackage{rotating}

\begin{document}

%%
%% The "title" command has an optional parameter,
%% allowing the author to define a "short title" to be used in page headers.
\title{Extensions to Separation Logic based tool chain in Coq}

%%
%% The "author" command and its associated commands are used to define
%% the authors and their affiliations.
%% Of note is the shared affiliation of the first two authors, and the
%% "authornote" and "authornotemark" commands
%% used to denote shared contribution to the research.


\author{Jordan Matthew Winkler}
\affiliation{%
 \institution{Purdue University}
 \city{West Lafayette}
 \state{Indiana}
 \country{USA}}
\email{jwinkler@purdue.edu}


%%
%% By default, the full list of authors will be used in the page
%% headers. Often, this list is too long, and will overlap
%% other information printed in the page headers. This command allows
%% the author to define a more concise list
%% of authors' names for this purpose.
\renewcommand{\shortauthors}{Jordan et al.}

%%
%% The abstract is a short summary of the work to be presented in the
%% article.
\begin{abstract}
\label{sec:abstract}

Coq is a programming language and theorem prover. It has many programming tools including the Verified Software Toolchain (VST).
The VST project is designed to bring software verification techniques to C, a popular memory managing language. Doing software verification with programming languages, that have memory management as part of their syntax,
is done using separation logic. This is due to the frame axiom that separation logic provides, allowing valid propositions invariant to the entire memory heap. There are many features to add to VST to extend its usefulness in software verification. One such feature that would be helpful is the ability to copy C structs, which is what I added to VST. I did this indirectly to VST, and am still working on a translation to a proven and C17 version.

\end{abstract}


%\begin{CCSXML}
%<ccs2012>
%   <concept>
%       <concept_id>10011007.10011006.10011073</concept_id>
%       <concept_desc>%Software and its engineering~Software maintenance tools
%       </concept_desc>
%       <concept_significance>500</concept_significance>
%       </concept>
% </ccs2012>
%\end{CCSXML}

%\ccsdesc[500]{Software and its engineering~Software maintenance tools}

%%
%% Keywords. The author(s) should pick words that accurately describe
%% the work being presented. Separate the keywords with commas.
\keywords{separation logic, calculus of inductive constructions, coq, vst}

%%
%% This command processes the author and affiliation and title
%% information and builds the first part of the formatted document.
\maketitle

\section{Introduction and Motivation}
\label{sec:introduction}


The practical benefits of being able to prove the correctness of C programs are in the usefulness of C programs in areas requiring high levels of software security and safety critical software. C is not like other garbage collector optional languages like Ada. C uses a lot of undefined behavior to make compilers easier to implement across hardware platforms. But C was designed to be used to port and maintain the Unix operating system. With the wide spread of Unix and Unix-like operating systems, and C's adoption into spaces that used to be reserved for assemblers, C is now a very common component in modern computer architecture.


In large programs, such as server file systems, bugs seem to take on an analytical property. I mean from the field of mathematical analysis. Given more time and effort, there always seems to be a new bug. It is as if software bugs continuously approach a neighborhood of zero bugs. But verification of software is appealing due to the potential for internal consistency of a program. Random acts of physics on the hardware might still be a concern, or maybe we proved the wrong problem correct and the customer is still unsatisfied. But it is an honest attempt at doing things correctly.

From “Social Processes and Proofs of Theorems and Programs" by Lipton and Perlis, they talked about the reality that mathematical proof is not as rigorous and reliable as people outside of the mathematical community like to believe it is. \cite{lipton_social_1979} And this is a valid point. But I feel that there are two aspects to mathematical culture. There is the art side of mathematics, that is cultural and human and has many rituals that would be lost in automated systems. And then there is the engineering side of mathematics, the part of mathematics that likes to think of itself as a science. This later aspect is what people use to build bridges and dams. It would be nice if this area of math is reliable.

I also wanted to learn a programming language that is good for both verification of software and new mathematical theories. The Curry-Howard isomorphism is an exciting and useful fact. And to consider the future of mathematical research, in mathematical subjects like algebraic topology, it seems harder each decade to generate correct proofs in some areas. Categories just have more computation involved. Mathematicians can stick to the sidelines of what can be pictured on a chalkboard. But I find that kind of work to be less interesting. There are new heights to be reached, possibly faster, with new technology. And one of the more appealing attributes of mathematical results obtained this way, is the gap to implementation of an algorithm with such a construct is much smaller.

 Working on the Verified Software Toolchain (VST) seemed like a good opportunity to learn many complicated concepts as they are focused on a particular problem. There is practical use of logics, programming language design, automated reasoning, and practice with a type theory in a proving language. VST has some interesting features. It is written in Coq, a common language for writing proofs and some very sturdy code. There are over 300 proven packages offered in the Coq language on Opam, and the standard library looks pretty nice for using more advanced mathematics with code. \cite{noauthor_opam_nodate} These external packages include SMT solvers and cubical type theory, and the core language library includes various number and set constructions. Being able to use Reals instead of floating point numbers might seem excessive, but Real numbers are the minimum complete set. A lot of analysis concepts only approximately work on something weaker. 
 
 VST itself includes libraries used to reason about manual memory management at the proof level. I was curious about this. There are also packages in VST like VST-Floyd, and FCF, and Compcert. VST-Floyd automates reasoning around separation logic. The Foundational Cryptography Framework FCF is a bunch of complicated yet proven algorithms for encryption. And Compcert is a compiler from C to x86, ARM, and PowerPC that also uses separation logic to do so. In retrospect I should have worked on Compcert, as it is one of the most documented Coq programs in the VST.

\section{Background and Related Work}
\label{sec:background_and_related_work}

Separation logic is an extension of Floyd–Hoare logic, that has been around since 2002. \cite{reynolds_separation_2002} There have been many innovations related to it since then. Oracle semantics for concurrent separation logic \cite{drossopoulou_oracle_2008}, separation algebra for a language called concurrent C minor \cite{hutchison_fresh_2009},  Multimodal, and higher order separation logics as well. \cite{dockins_multimodal_2008,noauthor_higher_nodate} Especially in the last 4 or so years, there has been enough progress in the subject to warrant a paper to organize this "jungle" of separation logics. \cite{chang_bringing_2017} Fairly recent implementations have made it into VST (Verified Tool Chain), which is coded in the Coq programming language and proof management system. \cite{hutchison_verified_2012} VST is a tool chain that is being developed mostly for the verification and compilation of the C programming language. VST has many modules, all of which use some separation logic to some extent. 

The first major chunk is a specification, a front end language called Verifiable C \cite{noauthor_verifiable_nodate}, which uses higher-order impredicative concurrent separation logic. Verifiable C works as a logical language that is mostly a restriction of the C language. Much like python 3 has what are called type hints \cite{noauthor_typing_nodate}, Verifiable C has constructs that help the underlying tool chain make more sense of C programs and verify their correctness. The next large section of code takes Verifiable C and checks for correctness. This part has parts called VST-FLoyd and a few other components. VST-Floyd is part of the core tactics of Verified C, and makes proving safety and correctness of C programs much more automatic. This section of code seems to have the ability to work outside of the C program verifying toolchain to also do separation logic operations as a Coq library. \cite{noauthor_princetonuniversityvst_2021}

On the back end is the last major chunk called CompCert, a C compiler that also uses separation logic in its compilation process. \cite{noauthor_compcert_nodate} This is not open source software and is owned by INRIA. It is a software project that is dual licensed. So I was not sure if contributions could be made to the software that would be welcome.

To understand this software stack I have found a series called Software Foundations. Software Foundations is a book series on solving, verifying, and some synthesis of programs with the Coq language. \cite{noauthor_software_nodate} There are 5 volumes covering these concepts. The most important for this project is volume 5: Verifiable C. There are also many manuals for parts of VST to go through, as needed. But it turns out the most useful way to learn VST and the Coq language was to look at the manuals for both languages. Software foundations is rated to take a semester for the first book, and potentially that long for each subsequent book. There was simply not enough time in the semester for this much reading.

\section{Work Timeline}
\label{sec:concrete strategy}

\subsection{Overview}

The work of adding a feature to a codebase is a fairly concrete process. In theory writing some code for VST is just a type-heavy type of programming. The issue is that there was a decent learning curve to trying to understand the Coq language, and how separation logic would work with a type based proof system. This was also a contribution to an existing project in a way that could not be separated into an individual module in an obvious way. In Coq, variables are immutable. So it was not possible to change the variable for the syntax tree without figuring out what function in the code base wrote it. And syntax transformations always seem to require many intermediate steps. I eventually decided to break the problem into stages that could be implemented and tested without covering the entire learning curve in one semester.

\subsection{Stage 1}

Stage 1 is to get syntax transformations working in any form. VST uses Clightgen to generate an abstract syntax tree of the C code in Coq, and the proof code needed to handle trivial and routine parts of the proof. I could attach the compiler on the front of the build process and have C compiler to C that VST can digest more easily. This would still be a useful tool in itself. The downside of this code is that it would not have been proven in any sort of proof system, and would be fairly unrelated to the various types of separation logic that are in VST. But,implimenting this process I used standard tests of the C syntax before and after being transformed. I also got a copy of the C17 standard and set out to work on this. C has significant amounts of undefined behavior for the sake of compiler writer freedom to make compilers portable and fast. But this also has the consequence that few programmers know what a canonical C bug is versus a quirk of an implementation. It was probably pedantic to be precise about a naturally vague if not imprecise language, versus something like Ada or Zig which are designed to not have undefined behavior. Ada in particular being a more natural choice for government software that needs this level of fault tolerance. But VST and Compcert are designed for standard C. So that is what I attempted to do too.

I originally started with C++ as the compiler language, but this turned out to be a poorly thought out language choice. It seemed natural at the time, as C++ is one of the most common languages to write compilers in, due to LLVM, Bison, Yacc, and other nice tools. Coq is more ML-like, and python as of 3.9 has types and homoiconicity needed for quicker manual testing. Coding in Python now is easier for a one person project, and porting from Python to Coq is also easier. I finished this stage. But there is a need for a lot more tests. And C, small as it is, still has a 500 page standard.

So, at the very least, I managed to create a tool to solve the problem of not being able to prove it is safe to copy structures in VST.

\subsection{Stage 2}

Stage 2 is to port the python code into working Coq code. I did not finish this stage. I wrote some Coq code to move simple trees around. But I am still working on getting more complex constructs to work. I liked that instead of conditional branching Coq primarily uses pattern matching. So the program is more like a search space definition. This is very aesthetically appealing from a computability point of view, where reasoning with search spaces instead of loops is common. With inductive types many data structures are also infinite in effective size. The downside was that Coq is kind of particular on what the pattern matching code is supposed to look like. So it was not as if I could give a before-and-after of the syntax tree and expect it to figure out the intermediate transformations. Translating all of the Python compiler to Coq may not really needed, as scanning and parsing are already done well enough by clightgen.

\subsection{Stage 3}

Stage 3 is to integrate the working Coq code into the project properly, and then do a full proof with the new feature. While testing VST, I found that clightgen will take C code with the transformation from the python compiler. I am not sure how to complete a proof yet though. As I will go into in the conclusions section, VST is not as well documented as I personally wanted. And this slows down my ability to understand what the system code is actually doing.

\section{Conclusions}
\label{sec:conclusions}

\subsection{Interesting Discoveries}

Looking at the limitations of the VST system for C code, such missing features may initially seem like mild feature exclusions. But these issues turn out to be common features of more hardware focused programming. Goto's, duff machines, and pointer conversions are well known features of C programs. Not having something like a conversion from the int data type to pointers of any kind points to a bigger hole in the proof system. VST is not currently able to reason about programs that do memory access without operating system assistance. For C code, in practice, this is a common feature. It points to an issue that, while the C code used with VST may be performing provably correct algorithms, if given the chance, C programs used are still going to need the usual set of tests in embedded systems. But I guess this would make sense. The kind of problems C is used to solve, outside of operating system design, have to deal with problems like signal noise from electronics. Compcert at the moment also only targets 3 families of architectures: ARM, x86, and PowerPC.

%Reading through the Verifiable C manual it seems like there are many clever things to handle C code
%One aspect that I learned about Coq that I found interesting, (was pretty much most of it, I am a fan of everything but the name)

\subsection{Road Blocks}

One of the larger issues with the project that made it hard to proceed was the lack of documentation, or incorrect documentation. Most of the code base in the clightgen portion of VST had little to no comments and single letter variable names. Take this code for example from $veric/Clight\_core.v$: \\

\begin{verbatim}
Axiom ef_deterministic_fun:
 forall ef,
  ef_deterministic ef = true ->
 forall  ge args m t1 t2 vres1 vres2 m1 m2,
  Events.external_call ef ge args m t1 vres1 m1 ->
  Events.external_call ef ge args m t2 vres2 m2 ->
  (vres1,t1,m1) = (vres2,t2,m2).
\end{verbatim}
\\

I can guess that this axiom is for an external function property. It should be a predicate that wraps around an external function callable, given that it is also a deterministic function. The way to see this, in theory, is that the same function $ef$, with the same global environment $ge$, argument list $args$, same global heap $m$, and the remaining parts of a separation logic triple (expression, stack, heap). The implication being that the external function will behave the same way on each function call.

I am happy that I can understand a structure like this, given some thought. And it is really cool to need to be fluent in separation logic to understand and work with some code. But there are thousands of such structures in the VST program. I am not really sure how to deal with proof code on a large scale yet.

It was not likely intentional to have incorrect documentation. Coq has facilities to control for versioning of the Coq compiler. This is done for the sake of safety, as Coq does not promise complete backwards compatibility with previous language definitions. But as a tool for verified code and proof system, it does try to promise soundness. 

Luckily there are three different sources of information to learn how to use the software by Appel et. al. \cite{hutchison_verified_2012,appel_program_2014,noauthor_verifiable_nodate-3} But all of these were about using the software, not how to modify it. 



\subsection{Potential Further Work}

It would be cool to see compcert ported to more computer architectures. The CompCert project at the moment is robust, but feature light. The documentation for it seems a lot fuller than with veriC and other parts of VST. So working on Compcert might be a more fruitful direction for learning proof coding techniques. Most of Compcerts C standard libraries are handled by the GCC compiler. VST handles situations like file permissions and concurrent programs. But, as far as I know, there is nothing yet to deal with Pipes and Sockets. It also does not have features for electronics interactions.

And I am sure the VST code base would make a lot more sense after I have read the proper manual, or if I got in contact with the software writers. It just was a lot to do and tough to know what to do. Given enough time and dedication I can achieve much more mastery of the Coq proof language and tools, and have a better intuition for what more proof code does. I would like to see if other features are easier to add at that point of improved competency.

I have not had time to use VST for a C project. The verifier tools are more on the algorithm end instead of exotic hardware interaction. So nothing has come to mind so far for a simple use case that a single programmer would write. For programs that do not explicitly need manual memory management I would be more apt to use Coq, Idris, or some other proving tool.
\section{Appendix B: Artifacts}
My code is version-controlled on GitHub. We provide a README and commented code 
at \url{https://github.com/Jmw150/Coq-VST-DLC}.


% \section{Introduction}
% ACM's consolidated article template, introduced in 2017, provides a
% consistent \LaTeX\ style for use across ACM publications, and
% incorporates accessibility and metadata-extraction functionality
% necessary for future Digital Library endeavors. Numerous ACM and
% SIG-specific \LaTeX\ templates have been examined, and their unique
% features incorporated into this single new template.

% If you are new to publishing with ACM, this document is a valuable
% guide to the process of preparing your work for publication. If you
% have published with ACM before, this document provides insight and
% instruction into more recent changes to the article template.

% The ``\verb|acmart|'' document class can be used to prepare articles
% for any ACM publication --- conference or journal, and for any stage
% of publication, from review to final ``camera-ready'' copy, to the
% author's own version, with {\itshape very} few changes to the source.

% \section{Template Overview}
% As noted in the introduction, the ``\verb|acmart|'' document class can
% be used to prepare many different kinds of documentation --- a
% double-blind initial submission of a full-length technical paper, a
% two-page SIGGRAPH Emerging Technologies abstract, a ``camera-ready''
% journal article, a SIGCHI Extended Abstract, and more --- all by
% selecting the appropriate {\itshape template style} and {\itshape
%   template parameters}.

% This document will explain the major features of the document
% class. For further information, the {\itshape \LaTeX\ User's Guide} is
% available from
% \url{https://www.acm.org/publications/proceedings-template}.

% \subsection{Template Styles}

% The primary parameter given to the ``\verb|acmart|'' document class is
% the {\itshape template style} which corresponds to the kind of publication
% or SIG publishing the work. This parameter is enclosed in square
% brackets and is a part of the {\verb|documentclass|} command:
% \begin{verbatim}
%   \documentclass[STYLE]{acmart}
% \end{verbatim}

% Journals use one of three template styles. All but three ACM journals
% use the {\verb|acmsmall|} template style:
% \begin{itemize}
% \item {\verb|acmsmall|}: The default journal template style.
% \item {\verb|acmlarge|}: Used by JOCCH and TAP.
% \item {\verb|acmtog|}: Used by TOG.
% \end{itemize}

% The majority of conference proceedings documentation will use the {\verb|acmconf|} template style.
% \begin{itemize}
% \item {\verb|acmconf|}: The default proceedings template style.
% \item{\verb|sigchi|}: Used for SIGCHI conference articles.
% \item{\verb|sigchi-a|}: Used for SIGCHI ``Extended Abstract'' articles.
% \item{\verb|sigplan|}: Used for SIGPLAN conference articles.
% \end{itemize}

% \subsection{Template Parameters}

% In addition to specifying the {\itshape template style} to be used in
% formatting your work, there are a number of {\itshape template parameters}
% which modify some part of the applied template style. A complete list
% of these parameters can be found in the {\itshape \LaTeX\ User's Guide.}

% Frequently-used parameters, or combinations of parameters, include:
% \begin{itemize}
% \item {\verb|anonymous,review|}: Suitable for a ``double-blind''
%   conference submission. Anonymizes the work and includes line
%   numbers. Use with the \verb|\acmSubmissionID| command to print the
%   submission's unique ID on each page of the work.
% \item{\verb|authorversion|}: Produces a version of the work suitable
%   for posting by the author.
% \item{\verb|screen|}: Produces colored hyperlinks.
% \end{itemize}

\printbibliography

\end{document}
